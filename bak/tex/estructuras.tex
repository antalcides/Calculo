
\chapter{Estructuras algebraicas}

\PartialToc

\hypersetup{linkcolor=ptctitle}

\vspace*{0cm}
 
\begin{flushright}
\textit{\scriptsize{}Los s�mbolos algebraicos se usan cuando no sabes
de qu� estas hablando. }
\par\end{flushright}{\scriptsize \par}

\begin{flushright}
\textbf{\textsc{\small{}Philippe Schnoebelen}}
\par\end{flushright}{\small \par}

\begin{flushright}
\textit{\footnotesize{}. }
\par\end{flushright}{\footnotesize \par}

\vspace*{-1mm}


\section{Introducci�n}

Las estructuras algebraicas constituyen una de las herramientas b�sicas
para tratar la mayor parte de los problemas asociados a la matem�tica
discreta. En este cap�tulo, pretendemos dar las primeras nociones
algebraicas para poder tratar, a t�tulo de ejemplo, algunos de los
problemas m�s conocidos dentro de este �mbito. El primer cap�tulo
se dieron los conceptos de relaciones, aplicaciones, operaciones y
ahora se presentan las estructuras algebraicas m�s importantes. Algunas
de las nociones que se consideran corresponden a una formaci�n b�sica,
pero por razones de coherencia formal se ha considerado conveniente
reconsiderarlas y agruparlas ordenadamente en su contexto original.
El cap�tulo siguiente est� dedicado al estudio de los conjuntos num�ricos
y los veremos como con ellos se pueden construir estructuras algebraicas
como modelos m�s completo de estructuras definidas a partir de una
operaci�n o dos operaciones. 


