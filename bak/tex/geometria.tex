
\chapter{Tópicos de geometría y trigonometría }

\PartialToc

\hypersetup{linkcolor=ptctitle}

\vspace*{0.5cm}
 
\begin{flushright}
\textit{\footnotesize{}El principio de razón suficiente, que afirma
que nada}
\par\end{flushright}{\footnotesize \par}

\begin{flushright}
\textit{\footnotesize{}sucede gratuitamente, es decir, que a todo
fenómeno }
\par\end{flushright}{\footnotesize \par}

\begin{flushright}
\textit{\footnotesize{}le corresponde una explicación, una razón de
ser }
\par\end{flushright}{\footnotesize \par}

\begin{flushright}
\textit{\footnotesize{}que se presente admisible a la razón. }
\par\end{flushright}{\footnotesize \par}

\begin{flushright}
{\small{} }Leibniz: 
\par\end{flushright}

\vspace*{-1mm}


\subsection{aaaaa}
