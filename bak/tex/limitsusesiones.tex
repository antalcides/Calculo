
\chapter{\hspace*{-5pt}Límites de sucesiones de números reales}

\PartialToc

\hypersetup{linkcolor=ptctitle}

\vspace*{0.5cm}
 
\begin{flushright}
\textit{\footnotesize{}El principio de razón suficiente, que afirma
que nada}
\par\end{flushright}{\footnotesize \par}

\begin{flushright}
\textit{\footnotesize{}sucede gratuitamente, es decir, que a todo
fenómeno }
\par\end{flushright}{\footnotesize \par}

\begin{flushright}
\textit{\footnotesize{}le corresponde una explicación, una razón de
ser }
\par\end{flushright}{\footnotesize \par}

\begin{flushright}
\textit{\footnotesize{}que se presente admisible a la razón. }
\par\end{flushright}{\footnotesize \par}

\begin{flushright}
{\small{} }Leibniz: 
\par\end{flushright}

\vspace*{-1mm}


\section{Introducción }

Existen multitud de problemas matemáticos que solo exigen para su
solución la consideración de conjuntos finitos de números, operaciones,
construcciones, etc. Tales problemas constituyen el objeto de la rama
de las Matemáticas que recibe el nombre de <<\textsf{Matemáticas
finita}>>.

Pero también se presentan problemas de gran interés teórico y práctico
, que requieren la consideración de conjuntos infinitos y que son
estudiados por una rama de las Matemáticas llamada \textsf{Cálculo
Infinitesimal}.

Entre los conjuntos infinitos uno de los más estudiados es el conjunto
de las \textsf{Sucesiones}. 
