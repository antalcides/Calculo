\newcommand{\R}{\ensuremath{\mathbb R}}
\newcommand{\Rp}{\ensuremath{{\mathbb R}^{+}}}
\newcommand{\Rm}{\ensuremath{{\mathbb R}^{-}}}
\newcommand{\Rpo}{\ensuremath{{\mathbb R}^{+}_{\,\rm o}}}
\newcommand{\Rmo}{\ensuremath{{\mathbb R}^{-}_{\,\rm o}}}
\newcommand{\Rn}{\ensuremath{{\mathbb R}^{n}}}
\newcommand{\Rx}[1]{\ensuremath{{\mathbb R}^{#1}}}
\newcommand{\N}{\ensuremath{\mathbb N}}
\newcommand{\Z}{\ensuremath{\mathbb Z}}
\newcommand{\nN}{\ensuremath{n\!\in\!\mathbb N}}
\newcommand{\Q}{\ensuremath{\mathbb Q}}
\newcommand{\C}{\ensuremath{\mathbb C}}
\newcommand{\K}{\ensuremath{\mathbb K}} % Cuerpo base
\newcommand{\cont}{\ensuremath{\mathcal{C}}}
\newcommand{\Riemann}{\ensuremath{\mathcal{R}}}%Funciones integrables %Riemann
\newcommand{\A}{\ensuremath{\mathcal{A}}}
\newcommand{\m}{\ensuremath{\Omega}}
\newcommand{\T}{\ensuremath{\mathcal{T}}}        % Topología
\newcommand{\ep}{\ensuremath{\varepsilon\in\Rp}}
\newcommand{\eps}{\ensuremath{\varepsilon}}
\newcommand{\epos}{\ensuremath{\varepsilon>0}}
\newcommand{\dpos}{\ensuremath{\delta>0}}
\newcommand{\irr}{\ensuremath{\mathbb R\!\setminus\!\mathbb Q}}
\newcommand{\ff}{\ensuremath{\varphi}}
\newcommand{\dis}{\displaystyle}
\newcommand{\funcreal}[2]{\ensuremath{\,#1\!:#2\rightarrow \mathbb R\,}}
\newcommand{\cfunc}[2]{\ensuremath{\,#1:#2\rightarrow \mathbb C\,}}
\newcommand{\suc}[1]{\ensuremath{\{#1\}}}
\newcommand{\sucsub}[2]{\ensuremath{\{#1_{#2}\}}}
\newcommand{\sucn}[1]{\ensuremath{\{#1_{n}\}}}
\newcommand{\sucm}[1]{\ensuremath{\{#1_{m}\}}}
\newcommand{\sucN}[2]{\ensuremath{\,\{#1_{#2}\}_{#2\in\mathbb N}}}
\newcommand{\sucNo}[2]{\ensuremath{\,\{#1_{#2}\}_{#2\in\mathbb N_{\rm o}}}}
\newcommand{\parcial}[1]{\ensuremath{\{#1_{\sigma(n)}\}}}
\newcommand{\parcialf}[1]{\ensuremath{\{#1_{varphi(n)}\}}}
\newcommand{\Suma}[1]{\mbox{${\displaystyle \sum_{#1=-\infty}^\infty}$}}
\newcommand{\suma}[1]{\mbox{${\displaystyle \sum_{#1}}$}}
\newcommand{\sumaf}[2]{\mbox{${\displaystyle \sum_{#1}^{#2}}$}}
\newcommand{\sumainf}[1]{\mbox{${\displaystyle \sum_{#1}^{\infty}}$}}
\newcommand{\sumasubn}[1]{\mbox{${\displaystyle \sum_{n\geqslant #1}}$}}
\newcommand{\serie}[1]{\mbox{${\displaystyle \sum_{n\geqslant 1}#1\,}$}}
\newcommand{\serien}[1]{\mbox{${\displaystyle \sum_{n\geqslant 1}\!#1_n\,}$}}
\newcommand{\serieno}[1]{\mbox{${\displaystyle \sum_{n\geqslant 0}\!#1\,}$}}
\newcommand{\Suman}[1]{\mbox{${\displaystyle \sum_{n=1}^{\infty}\!#1_n\,}$}}
\newcommand{\ser}[1]{\mbox{$\{#1_1+#1_2+\cdots+#1_n\!\}$}}
\newcommand{\serabs}[1]{\mbox{$\{|#1_1|+|#1_2|+\cdots+|#1_n|\!\}$}}
\newcommand{\armd}{\mbox{$\dis\bigg\{1+\frac{1}{2}+\cdots+\frac{1}{n}\bigg\}$}}
\newcommand{\arm}{\mbox{$\{1+1/2+\cdots+1/n\}$}}
\newcommand{\Arm}{\mbox{$1+\dfrac{1}{2}+\cdots+\dfrac{1}{n}$}}
\renewcommand{\en}{\!\in\!}
\newcommand{\bcapn}[1]{\mbox{$\dis\bigcap_{#1\in\mathbb N}$}}
\newcommand{\bcupn}[1]{\mbox{$\dis\bigcup_{#1\in\mathbb N}$}}
\newcommand{\bcupzn}[1]{\mbox{$\dis\bigcup_{#1\in\mathbb Z}$}}
\newcommand{\integ}[3]{\mbox{$\int_{#1}^{#2}#3\,$}}
\newcommand{\dispinteg}[3]{\mbox{${\displaystyle \int_{#1}^{#2}\!#3\,}$}}
\newcommand{\vac}{$\emptyset$}
\newcommand{\fin}{~\hfill $\Box$\newline\smallskip}
\newcommand{\sskip}{\vspace{2mm}}
\newcommand{\bskip}{\vspace{3mm}}
\newcommand{\dem}{\noindent{\textbf{\textcolor{myblue!75!black}{Demostraci\'on}}.\ }}
\newcommand{\sol}{\noindent{\textbf{\textcolor{myblue!75!black}{Soluci\'on.}}\ }}
\newcommand{\hecho}{\noindent{~\hfill\Large\Smiley{} }}% usa marvosym
\DeclareMathOperator{\dist}{dist}
\DeclareMathOperator{\sen}{sen}
\DeclareMathOperator{\tg}{tg}
\DeclareMathOperator{\arctg}{arctg}
\DeclareMathOperator{\arcsen}{arcsen}
\DeclareMathOperator{\senh}{senh}
\DeclareMathOperator{\argsenh}{argsenh}
\DeclareMathOperator{\argcosh}{argcosh}
\DeclareMathOperator{\Arccos}{Arccos}
\DeclareMathOperator{\Arcsen}{Arcsen}
\DeclareMathOperator{\Arctg}{Arctg}
\DeclareMathOperator{\cotg}{cotg}
\DeclareMathOperator{\cosec}{cosec}
\DeclareMathOperator{\tgh}{tgh}
\DeclareMathOperator{\argtgh}{argtgh}
\DeclareMathOperator{\argcosech}{argcosech}
\DeclareMathOperator{\Arg}{Arg}
\DeclareMathOperator{\arcsec}{arcsec}
\DeclareMathOperator{\arccosec}{arccosec}
\DeclareMathOperator{\argsech}{argsech}
\newcommand{\enC}[1]{\ensuremath{#1\!\in\!\mathbb C}}
%\DeclareMathOperator{\re}{Re}
\DeclareMathOperator{\im}{Im}
\DeclareMathOperator{\Log}{Log}
\newcommand{\conj}[1]{\mbox{$\overline{\rule{0mm}{1.8mm}#1}$}}
\newcommand{\Lim}[3]{\mbox{$\displaystyle{\lim_{#2\to #3}#1}$}}
\newcommand{\limlft}[3]{\mbox{$\dis{\lim_{\substack{#2\to #3\\ #2\,<\,#3}}#1}$}}
\newcommand{\limrgt}[3]{\mbox{$\dis{\lim_{\substack{#2\to #3\\ #2\,>\,#3}}#1}$}}
\newcommand{\Dfa}{\mbox{$f^{\,\prime}(a)$}}
\newcommand{\Dfka}{\mbox{$f^{\,(k)}(a)$}}
\newcommand{\Dfna}{\mbox{$f^{\,(n)}(a)$}}
\DeclareMathOperator{\e}{e}
\newcommand{\tl}{\mbox{$^{\mspace{2mu}\prime}$}}
\newcommand{\tlo}{\mbox{$^{\prime}$}}
\newcommand{\fder}[1]{\mbox{$#1^{\,\prime}$}}
\newcommand{\Derdos}[2]{\mbox{$#1^{\,\prime\prime}(#2)$}}
\newcommand{\derpar}[2]{\mbox{$\dfrac{\partial #1}{\partial #2}$}}
\newcommand{\derpardos}[2]{\mbox{$\dfrac{\partial^{2}\! #1}{\partial #2^{2}}$}}
\newcommand{\scd}{\mbox{$^{\,\prime\prime}$}}
\newcommand{\Der}[2]{\mbox{$#1^{\,\prime}(#2)$}}
\newcommand{\Derk}[2]{\mbox{$#1^{\,(k)}(#2)$}}
\newcommand{\Dern}[2]{\mbox{$#1^{\,(n)}(#2)$}}
\newcommand{\escalar}[2]{\ensuremath{\left\langle #1\,\big |\, #2\right\rangle}}
\newcommand{\norm}[1]{\ensuremath{\lVert#1\rVert}}
\newcommand{\partx}[1]{\ensuremath{\dfrac{\partial #1}{\partial x}}}
\newcommand{\party}[1]{\ensuremath{\dfrac{\partial #1}{\partial y}}}
\newcommand{\partz}[1]{\ensuremath{\dfrac{\partial #1}{\partial z}}}
\newcommand{\partxx}[1]{\ensuremath{\dfrac{\partial^2 #1}{\partial x^{2}}}}
\newcommand{\partyy}[1]{\ensuremath{\dfrac{\partial^2 #1}{\partial y^2}}}
\newcommand{\partxy}[1]{\ensuremath{\dfrac{\partial^2 #1}{\partial x
\partial y}}}
\newcommand{\partyx}[1]{\ensuremath{\frac{\partial^2 #1}{\partial y
\partial x}}}
\newcommand{\partone}[2]{\ensuremath{\dfrac{\partial #1}{\partial #2}}}
\newcommand{\partwo}[2]{\ensuremath{\dfrac{\partial^2 #1}{\partial #2^2}}}
\newcommand{\partonetwo}[3]{\ensuremath{\dfrac{\partial^2 #1}{\partial #2\partial #3}}}
\newcommand{\setbig}[1]{\big\{ #1 \big\}}
\newcommand{\set}[1]{\left\lbrace #1 \right\rbrace}
\renewcommand{\ge}{\geqslant}
\renewcommand{\le}{\leqslant}
%\newcommand{\inte}[1]{\overset{\circ}{#1}}      % Interior de un conjunto
\newcommand{\cajadoble}{\psdblframebox[linewidth=.6pt]}
\newcommand{\azul}{\color[rgb]{0,0,1}}
\newcommand{\negro}{\color[rgb]{0,0,0}}
\newcommand{\rojo}{\color[rgb]{1,0,0}}
\newcommand{\lra}[1]{\ensuremath{\left\langle #1 \right\rangle}}
\newcommand{\modulo}[1]{\left\lvert{#1}\right\rvert}
\newcommand{\abs}[1]{\lvert{#1}\rvert}  % Valor absoluto
\newcommand{\norma}[1]{\left\|{#1}\right\|}         % Norma
\newcommand{\df}[1]{\,\mathrm{d}#1\,}               % Diferencial
\newcommand{\derivada}[2]{\dfrac{\df{#1}}{\df{#2}}}  % Derivada de #1 respecto de #2
\newcommand{\derivadados}[2]{\ensuremath{\dfrac{\mathrm{d}^2 #1}{\mathrm{d}#2^2}}}
\newcommand{\derivadatres}[2]{\ensuremath{\dfrac{\mathrm{d}^3 #1}{\mathrm{d}#2^3}}}
\newcommand{\derivadan}[2]{\ensuremath{\dfrac{\mathrm{d}^n #1}{\mathrm{d}#2^n}}}
\newcommand{\derparcial}[2]{\ensuremath{\dfrac{\partial #1}{\partial
#2}}}
\DeclareMathOperator{\fr}{Fr}         % Frontera
\renewcommand{\leq}{\leqslant}
\renewcommand{\geq}{\geqslant}
\newcommand{\mb}{\mathbf}
\DeclareMathOperator{\senc}{senc}
\DeclareMathOperator{\sgn}{sgn}
\renewcommand{\Re}{\re}
\renewcommand{\Im}{\im}
\newcommand{\ms}{\mspace}
\newcommand{\msi}{\mspace{1mu}}
\newcommand{\msii}{\mspace{2mu}}
\newcommand{\msiii}{\mspace{3mu}}
\newcommand{\into}{\int_0^\infty}
\newcommand{\nw}[1]{\ensuremath{\overset{\centerdot}{#1}}}
\newcommand{\nww}[1]{\ensuremath{\overset{\centerdot\centerdot}{#1}}}
\newcommand{\func}[2]{\mbox{$\,#1:#2\rightarrow \R\,$}}
\newcommand{\vectorx}{\ensuremath{(x_1,x_2,\dots,x_n)}}
\newcommand{\vectory}{\ensuremath{(y_1,y_2,\dots,y_n)}}
\newcommand{\vectorz}{\ensuremath{(z_1,z_2,\dots,z_n)}}
\newcommand{\vectora}{\ensuremath{(a_1,a_2,\dots,a_n)}}
\newcommand{\vectorb}{\ensuremath{(b_1,b_2,\dots,b_n)}}
\newcommand{\vectorc}{\ensuremath{(c_1,c_2,\dots,c_n)}}
\newcommand{\esc}[2]{\ensuremath{\left\langle #1\, \mathbf{|}\,#2 \right\rangle}}
\newcommand{\vx}{\ensuremath{\mathbf{x}}}
\newcommand{\vy}{\ensuremath{\mathbf{y}}}
\newcommand{\vz}{\ensuremath{\mathbf{z}}}
\newcommand{\va}{\ensuremath{\mathbf{a}}}
\newcommand{\vb}{\ensuremath{\mathbf{b}}}
\newcommand{\vc}{\ensuremath{\mathbf{c}}}
\newcommand{\vu}{\ensuremath{\mathbf{u}}}
\newcommand{\vv}{\ensuremath{\mathbf{v}}}
\newcommand{\vw}{\ensuremath{\mathbf{w}}}
\newcommand{\vr}{\ensuremath{\mathbf{r}}}
\newcommand{\vf}{\ensuremath{\mathbf{F}}}
\newcommand{\vn}{\ensuremath{\mathbf{n}}}
\newcommand{\vt}{\ensuremath{\mathbf{t}}}
\newfont{\esclar}{cmmib10}% El punto para el producto escalar
\newcommand{\dt}{\esclar\mbox{\symbol{58}}}
\newfont{\mifuente}{cmbsy10}% El infinito es  el carácter 49 en esta fuente.
\newcommand{\infinity}{\mifuente\mbox{\symbol{49}}} %Símbolo infinito
\newfont{\adornos}{fourier-orns}% ornamentos
\newcommand{\volutaleft}{\adornos\mbox{\symbol{91}}} % adorno voluta
\newcommand{\volutaright}{\adornos\mbox{\symbol{92}}} % adorno voluta
\newcommand{\florleft}{\adornos\mbox{\symbol{98}}} % adorno flor
\newcommand{\florright}{\adornos\mbox{\symbol{99}}} % adorno flor
\newcommand{\manoright}{\adornos\mbox{\symbol{116}}} % adorno mano izquierda
\newcommand{\manoleft}{\adornos\mbox{\symbol{117}}} % adorno mano derecha
\newfont{\librotex}{manfnt}% figuras del libto de Knuth "The TeX Book"
\newcommand{\curvasr}{\librotex\mbox{\symbol{126}}} % curvas peligrosas dcha
\newcommand{\curvasl}{\librotex\mbox{\symbol{127}}} % curvas peligrosas izqda
\newfont{\noesta}{psyr}
\renewcommand{\notin}{\noesta\mbox{\symbol{207}}}
%\newfont{\implicadcha}{mtsyt}
\renewcommand{\Longrightarrow}{\implicadcha\mbox{\symbol{144}}}
\newcommand{\raiz}{\mbox{$\sqrt{a x^{2} + b x + c}\,$}}
\newcommand{\rac}[1]{\mbox{$\sqrt{#1}\,$}}
%\newfont{\wasi}{wasy10}% para incluir integrales rectas
%\newcommand{\varint}{\wasi\mbox{\symbol{119}}} %NO funciona
%\newcommand{\rac}[1]{\mbox{$\sqrt{\rule{0mm}{3.2mm} #1}\,$}}
%\newcommand{\raiz}{\mbox{$\sqrt{\rule{0mm}{3.2mm} ax^{\,2}+bx+c}\,$}}
%\newfont{\noestados}{usyr}
%\renewcommand{\notin}{\noestados\mbox{\symbol{207}}}
%\newfont{\sonrisa}{wasy10}% para incluir el smiley
%\newcommand{\smiley}{\sonrisa\mbox{\symbol{44}}}
%\newfont{\sonrisagrande}{umvs}% fuente marvosym
%\newcommand{\smileygrande}{\sonrisagrande\mbox{\Large\symbol{169}}}
%\renewcommand{\Longrightarrow}{\largaimplica\mbox{\symbol{222}}}
% Espacio de las funciones %continuas
%\newcommand{\h}{\ensuremath{\mathcal{H}}}        % Funciones holomorfas
%\newcommand{\M}{\ensuremath{\mathcal{M}}}   % Transformaciones de Möbius
%\newcommand{\F}{\ensuremath{\mathcal{F}}}
%\newcommand{\Cc}{\ensuremath{\mathscr{C}}}       % Conjunto de %rectas-circunferencias
%\newcommand{\Ca}{\ensuremath{\widehat{\mathbb C}}} % El cuerpo de los %complejos %ampliado con el infinito %\DeclareMathOperator{\rot}{rot} \DeclareMathOperator{\diver}{div}
%\renewcommand{\pi}{\,\piup} % requiere cargar txfonts
%\newcommand{\tf}[1]{\ensuremath{\mathscr{F}\!#1}}
%\newcommand{\tfi}[1]{\ensuremath{\mathscr{F}^{\,-1}\!#1}}
%\newcommand{\itf}[1]{\ensuremath{\check{#1}}}
%\newcommand{\tla}[1]{\ensuremath{\mathscr{L}\!#1}}
%\newcommand{\tli}[1]{\ensuremath{\mathscr{L}^{\,-1}\!#1}} %\DeclareMathOperator{\res}{Res}
%\DeclareMathOperator{\ind}{Ind}% Índice respecto de una curva
%\renewcommand{\emptyset}{\vac}  % Para redefinir el símbolo del conjunto vacío
%\renewcommand{\int}{\varint}    % para usar el símbolo de integral del paquete %wasysym %\newcommand{\csuma}{\overset{\centerdot}{+}}        % Yuxtaposición de curvas
%\newcommand{\copuesta}{\overset{\centerdot}{-}}     % Curva opuesta
%\renewcommand{\blacksquare}{\framebox[3.3mm][c]{\checkmark}} %\newcommand{\derparcial}[2]{\dfrac{\partial #1}{\partial #2}}
%                                                    % Derivada parcial de #1 %respecto #2

%\newcommand{\derivada}[2]{\ensuremath{\dfrac{\mathrm{d}#1}{\mathrm{d\,}#2}}} %\newcommand{\intr}{\int_{-\infty}^\infty \!\!\!}
%\newcommand{\sha}{\ensuremath{\textmd{I}\mspace{-1.5mu}\textmd{I}\mspace{-1.5mu} \textmd{I}}}
%\definecolor{naranja}{cmyk}{0,0.61,0.87,0}
%\definecolor{melocoton}{cmyk}{0,0.32,0.52,0}
%\definecolor{bronce}{cmyk}{0,0.85,0.87,0.35}
%\definecolor{azulclaro}{rgb}{0.8,0.85,1}
%\definecolor{melocoton}{rgb}{1,0.34,0.123}
%\definecolor{grisclaro}{rgb}{0.972503,0.972503,1}
%\definecolor{melon}{rgb}{0.889996,0.659993,0.410001}
%\definecolor{amarilloclaro}{rgb}{1,1,0.878399}
%\definecolor{amarilloclarodos}{rgb}{1, 1, 0.878399}
%\newcommand{\abrircerrar}[1]{\begin{center}\fboxrule0.7pt\fboxsep6pt\fcolorbox{blue}{amarilloclaro}{\begin{minipage}[c]{13.5cm}\bigskip #1 \smallskip\end{minipage}}\fboxrule0.4pt\fboxsep3pt\end{center}}
%\newcommand{\largeabrircerrar}[1]{\begin{center}\fboxrule0.7pt\fboxsep6pt\fcolorbox{blue}{amarilloclaro}{\begin{minipage}[c]{14cm}\bigskip #1 \smallskip\end{minipage}}\fboxrule0.4pt\fboxsep3pt\end{center}}
%\newcommand{\cajadobleroja}{\psdblframebox[linewidth=.6pt,linecolor=red]} %\newcommand{\Fr}[2][]{\fr_{#1}{#2}}           % Frontera de un conjunto
%\newcommand{\Sp}{\ensuremath{\mathbb{S}}}          % Esfera %\newcommand{\solb}{\noindent{\textbf{\color{Blue}Solución.\color{Black}}}\newline\noindent}
%\newcommand{\hecho}{\noindent{~\hfill\smiley}} %\newcommand{\raiz}{\mbox{$\sqrt{\rule{0mm}{3.2mm} ax^{\,2}+bx+c}\,$}}
%\newcommand{\q}{\mbox{$^{\,2}$}}
%\newcommand{\rac}[1]{\mbox{$\sqrt{\rule{0mm}{3.2mm} #1}\,$}}
%\newcommand{\defi}{\overset{\textrm{def}}{=}} %\newcommand{\lineinteg}[2]{\mbox{${\displaystyle \int_{#1}\!#2\,}$}}
%\newcommand{\Ind}[1]{\mbox{${\rm Ind}_{#1}$}} %\newcommand{\demb}{\noindent\color{Blue}\textbf{Demostración.}\color{Black}\ %}
%\DeclareMathOperator{\Res}{Res} %\newcommand{\modfz}[1][f]{\mbox{$\vert #1(z)\vert $}}
%\newcommand{\fh}[1]{\mbox{$f\en{\cal H}(#1)$}} %\newcommand{\finb}{~\hfill\color{Blue}$\blacksquare$\color{Black}\newline\medskip} %\newcommand{\fhol}[1]{\mbox{$f\en{\mathcal H}(#1)$}}
%\newcommand{\fol}{\mbox{$f\en{\mathcal H}(\Omega)$}}
%\newcommand{\foldisc}{\mbox{$f\en{\mathcal H}(D(0,1))$}}
%\newcommand{\folc}{\mbox{$f\en{\mathcal H}({\mathbb C})$}} 