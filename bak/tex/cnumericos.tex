
\chapter{Conjuntos numéricos}

\PartialToc

\hypersetup{linkcolor=ptctitle}

\vspace*{0.5cm}
 
\begin{flushright}
\textit{\footnotesize{}El principio de razón suficiente, que afirma
que nada}
\par\end{flushright}{\footnotesize \par}

\begin{flushright}
\textit{\footnotesize{}sucede gratuitamente, es decir, que a todo
fenómeno }
\par\end{flushright}{\footnotesize \par}

\begin{flushright}
\textit{\footnotesize{}le corresponde una explicación, una razón de
ser }
\par\end{flushright}{\footnotesize \par}

\begin{flushright}
\textit{\footnotesize{}que se presente admisible a la razón. }
\par\end{flushright}{\footnotesize \par}

\begin{flushright}
{\small{} }Leibniz: 
\par\end{flushright}

\vspace*{-1mm}


\section{Introducción }

En la antigüedad, el concepto de número surgió como consecuencia de
la necesidad práctica de contar objetos. Para ello, al principio el
hombre se valió de los elementos de que disponía a su alrededor: dedos,
piedras... Basta recordar, por ejemplo, que la palabra cálculo deriva
de la palabra latina \texttt{calculus}, que significa <<\texttt{contar
con piedras}>>. La serie de números naturales era, obviamente, limitada;
pero la conciencia sobre la necesidad de ampliar el conjunto de los
números naturales, representaba ya una importante etapa en el camino
hacia la matemática moderna. Paralelamente a la ampliación de los
conjuntos numéricos, se desarrollaron su simbología y los sistemas
de numeración diferentes para cada civilización.

La operación de contar exige la adición sucesiva del número 1 consigo
mismo, una y otra vez. Los números obtenidos de esta manera son llamados
números naturales o números enteros positivos. Con el progreso de
la técnica aparecen problemas de medición más complicados que requieren
la división de números naturales por que se construyen así los números
racionales positivos, después es necesario introducir un número $0$
llamado cero, con la propiedad de que todo número $a$, $a+0=a.$
Más tarde, cuando aparece el problema de distinguir entre lo que se
tiene y lo que no, se hace necesario introducir que para todo número
$a$ existe un número $-a$ llamado opuesto o negativo que cumple
que: $a+\left(-a\right)=0.$ Esto nos lleva a construir un nuevo conjunto,
el conjunto de los números negativos, o enteros negativos, de esta
manera se llega al concepto de número que conocemos, con los cuales
se pueden realizarlas operaciones de adición, multiplicación, substracción
y división. Este concepto comprende a los números racionales y a otros
números llamados irracionales.

Otras operaciones pueden ser definidas a partir de las anteriores,
como son, la potenciación, la radicación y otras más, además que la
substracción y la división pueden definirse a partir de la adición
y la multiplicación y que la mayoría de las leyes que rigen a los
números pueden ser demostradas a partir de un pequeño número de leyes
básicas o fundamentales. Por lo tanto vamos a establecer un conjunto
de leyes a partir de las cuales demostraremos el resto.

\section{Números naturales}

\subsection{Introducción}

\textbf{El cero ¿es o no un número natural? }

Este es uno de los temas de más frecuente discusión entre quienes
se dedican a las matemáticas. 

Cuando \texttt{Peano} introdujo los axiomas para definir el conjunto
de los números naturales, inició este conjunto por el número uno.
Pero cuando Cantor estudió la teoría de conjuntos, encontró que debía
empezar por el cero, dada la necesidad de asignarle un cardinal al
conjunto vacío. Quizá fue esto lo que hizo que, diez años más tarde,
\texttt{Peano} empezara los números naturales con el cero.

En las últimas décadas ha sido muy popular la teoría de conjuntos,
lo cual justifica que muchos profesores prefieran comenzar el conjunto
de los números naturales por el cero. En este capítulo se elige iniciar
el conjunto de los números naturales por el uno, aunque el cero es
necesario para el cardinal del conjunto vacío, para el neutro de la
suma y para tantas otras aplicaciones. Pero en algunos temas, como
el de las sucesiones, es mejor iniciar los números naturales sin el
cero, pues es normal que se relacione el primer término con el número
uno, el segundo término con el número dos y así sucesivamente, recordando
que no hay un ordinal para el cardinal cero. 

\textbf{Axiomas de Peano}

La más conocida axiomatización de los números naturales, contenida
en el escrito \textbf{\textsl{Arithmetices Principia Nova Methodo
Exposita}} del italiano \textbf{\textsl{Giuseppe Peano}}, se presenta
en esta sección en forma detallada, al igual que la forma mas moderna
de la axiomatización, la definición de las operaciones y sus propiedades
debidamente demostradas. 

Durante el siglo XlX, sin duda con el impulso de la aparición de las
geometrías no euclidianas, se multiplicaron los esfuerzos por axiomatizar
la geometría empeño culminado finalmente en 1899 con la publicación
de Fundamentos de la Geometría, de \texttt{David Hilbert}, también
se presentaron varias axiomatizaciones de la aritmética o, con más
precisión, de los números naturales. 

Sin lugar a dudas, la más conocida es la que presentó el matemático
italiano Giuseppe Peano (1858-1932) por primera vez en 1889 en un
pequeño libro publicado en Turín titulado \textbf{\textsl{Arithmetices
Principia Nova Methodo Exposita}}. Este texto incluye sus famosos
axiomas, pero más que un texto de aritmética, este documento contiene
una introducción a la lógica en la cual se presentan por primera vez
los símbolos actuales para representar la pertenencia, la existencia,
la contenencia (en la actualidad es invertido, acorde con el de los
números) y para la unión y la intersección. 

\texttt{Peano} reconoce hacer uso de estudios de otros autores: en
1888 después de estudiar a \texttt{G. Boole}, \texttt{E. Schröder},
\texttt{C. S. Peirce} y otros, estableció una analogía entre operaciones
geométricas y algebraicas con las operaciones de la lógica; en aritmética
menciona el trabajo de \texttt{Dedekind} publicado el año anterior
reconocido de manera generalizada como la primera axiomatización de
la aritmética, aunque salió a la luz 7 años después del artículo de
\texttt{Peirce} y un texto de \texttt{Grassmann} de 1861. Este ultimo
libro posiblemente fue fuente de inspiración tanto para \texttt{Peano}
y \texttt{Dedekind} como para \texttt{Peirce}. 

\texttt{Arithmetices Principia}, escrito en latín es el primer intento
de \texttt{Peano}, para lograr una axiomatización de las matemáticas
en un lenguaje simbólico. 

Consiste en un prefacio y 10 secciones:

1. Números y Adición.

2. Sustracción.

3. Máximos y Mínimos. 

4. Multiplicación.

5. Potenciación. 

6. División.

7. Teoremas varios.

8. Razones de Números.

9. Sistemas de Racionales e Irracionales.

10. Sistemas de Cantidades.

En libro establece los siguientes axiomas:
\begin{enumerate}
\item $1\in\na$.
\item Si $a\in\na$entonces $a=a.$
\item Si $a\in\na$ entonces $a=b$ si y sólo si $b=a.$
\item Si $a,b,c\in\na$ entonces $a=b,$ $b=c$ implica $a=c.$
\item Si $a=b$ y $b\in\na$ entonces $a\in\na.$
\item Si $a\in\na$ entonces $a+1\in\na.$
\item Si $a\in\na$ entonces $a=b$ si y sólo si $a+1=b+1$.
\item Si $a\in\na$ entonces $a+1\neq1.$
\item Si $k$ es una clase, $1\in k,$ y si para $x\in\na:$ $x\in k$ implica
$\left(x+1\right)\in k$.
\end{enumerate}

\subsection{Axiomas de los números naturales}

Empezaremos ahora la construcción de los números naturales siguiendo
la idea de Peano.\medskip{}
\medskip{}

\begin{axioma}{Existencia de los números naturales\label{ap1}}  
Existe un conjunto $\na\neq\emptyset$ llamado conjunto de los números
Naturales.

\end{axioma}

A los elementos de $\na$ se les llama números naturales, este axioma
no establece sus características ni sus propiedades, por lo que hay
que establecer otros axiomas para entender el concepto de número.

\vspace*{20pt}

\begin{axioma}{Existencia del número mínimo\label{ap4}}

Existe un número llamado $1\in\na.$

\end{axioma}

Del axioma \myref{ap1} tenemos que como $\na\neq\emptyset$ entonces
debe tener por lo menos un elemento, el axioma \myref{ap4} lo define
como el número $1.$

\medskip{}
\medskip{}

\begin{axioma}{Relación de igualdad\label{ap3}}

Existe una relación $"="$ tal que $a=b$ significa que $a$ y $b$
representan el mismo número natural, que cumpla las siguientes propiedades

$\forall a,b,c\in\na$ se tiene que:
\begin{description}
\item [{i)}] Si $a\in\na$entonces $a=a.$
\item [{ii)}] Si $a\in\na$ entonces $a=b$ si y sólo si $b=a.$
\item [{iii)}] Si $a,b,c\in\na$ entonces $a=b,$ $b=c$ implica $a=c.$
\item [{iv)}] Si $a=b$ y $b\in\na$ entonces $a\in\na.$
\end{description}
\end{axioma}

El concepto de igualdad \texttt{Peano} lo presentó como un axioma
y además establece que es una relación.

Observamos también que generalmente cuando se define la igualdad solo
se establecen las tres primeras propiedades, y la propiedad iv) es
nueva, la cual nos asegura que un número natural sólo puede ser representado
por un número natural. Es decir que si existen otros números estos
no pueden representar a un número natural.\medskip{}
\medskip{}

\begin{axioma}{Sucesor\label{ap2}}

Existe una aplicación %
\begin{tabular}{cccc}
$\varphi$ : &
$\na$ &
$\rightarrow$ &
$\na$\tabularnewline
\end{tabular} que cumple las siguientes propiedades
\begin{description}
\item [{i)}] $\forall n\in\na,\;\varphi\left(n\right)\neq1$
\item [{ii)}] $\forall n,m\in\na\quad\varphi\left(n\right)=\varphi\left(m\right)\Rightarrow n=m.$
\end{description}
\end{axioma}

La propiedad i) que establece el axioma \myref{ap2} nos dice que
el número uno no tiene preimagen, es decir no es sucesor de ningún
natural, es el primero.

La segunda propiedad del axioma \myref{ap2} establece la relación
de igualdad, en función de la función sucesor, pero también nos aclara
que el sucesor de un número natural es único. \medskip{}
\medskip{}

\begin{axioma}{Adición entre naturales\label{ap5}}

Existe una operación %
\begin{tabular}{cccc}
$+$ : &
$\na\times\na$ &
$\rightarrow$ &
$\na$\tabularnewline
 &
$\left(a,b\right)$ &
$\mapsto$ &
$c=+\left(a,b\right)=a+b$\tabularnewline
\end{tabular} que cumple las siguientes condiciones .

$\forall n,n\in\na$ se tiene que 
\begin{description}
\item [{i)}] $n+1=\varphi\left(n\right)$
\item [{ii)}] $\varphi\left(n\right)+m=n+\varphi\left(m\right)=\varphi\left(n+m\right)$
\end{description}
\end{axioma}

Hubiéramos podido establecer los axiomas \myref{ap1} y \myref{ap5}
como definiciones, pero decidimos establecerlos como axiomas, porque
no conocemos todavía la estructura de $\na$.

\begin{defi}{Sucesor}{sucesor}

La aplicación que garantiza el axioma \myref{ap2} la definiremos
\begin{tabular}{cccc}
$\varphi$ : &
$\na$ &
$\rightarrow$ &
$\na$\tabularnewline
 &
$n$ &
$\mapsto$ &
$\varphi\left(n\right)=n+1$\tabularnewline
\end{tabular} de acuerdo con la propiedad i) del axioma \myref{ap5}

\end{defi}

Esta definición la podemos expresar de la siguiente manera:
\[
\mbox{Si \ensuremath{n\in\na\Rightarrow n+1\in\na.}}
\]

Y con el podemos construir los números naturales de la siguiente forma
\begin{description}
\item [{i)}] $1\in\na$ entonces existe $\varphi\left(1\right)=1+1\in\na$.
Al número $1+1$ lo llamaremos dos, es decir $1+1:=2.$
\item [{ii)}] $2\in\na$ entonces existe $\varphi\left(2\right)=2+1\in\na.$
Al números $2+1$ lo llamaremos tres, es decir $2+1:=3.$
\item [{iii)}] $3\in\na$ entonces existe $\varphi\left(3\right)=3+1\in\na.$
Al números $3+1$ lo llamaremos cuatro, es decir $3+1:=4.$
\end{description}
Podemos seguir construyendo los números naturales de manera indefinida,
es decir el conjunto de los números naturales es infinito. 

En estos momentos como ya conocemos la estructura del conjunto de
los números naturales podemos definir la adición entre naturales de
acuerdo con la definición \ref{df:sucesor}.

Sean $m$ y $n$ números naturales entonces $m+n=\underset{m\mbox{ veces}}{(1+1+1+\cdots+1)}+\underset{n\mbox{ veces}}{\left(1+1+1+\cdots+1\right)}$
es decir:

\begin{table}[H]
\centering

\caption{Tabla de la adición entre números naturales.}
\resizebox{0.3\hsize}{!}{

\setlength\arrayrulewidth{1pt}\arrayrulecolor{ptctitle} 

\begin{tabular}{c|ccccc}
\arrayrulecolor{ptctitle}\hline\cellcolor{ptctitle!50}$+$ &
\cellcolor{ptctitle!50}$1$ &
\cellcolor{ptctitle!50}$2$ &
\cellcolor{ptctitle!50}$3$ &
\cellcolor{ptctitle!50}$4$ &
\cellcolor{ptctitle!50}$\cdots$\tabularnewline
\hline 
\hline\cellcolor{ptctitle!50}$1$ &
\cellcolor{ptcbackground} $2$ &
\cellcolor{ptcbackground} $3$ &
\cellcolor{ptcbackground}$4$ &
\cellcolor{ptcbackground}$5$ &
\cellcolor{ptcbackground}$\cdots$\tabularnewline
\hline\cellcolor{ptctitle!50}$2$ &
\cellcolor{gray!50} $3$ &
\cellcolor{gray!50} $4$ &
\cellcolor{gray!50}$5$ &
\cellcolor{gray!50}$6$ &
\cellcolor{gray!50}$\cdots$\tabularnewline
\hline\cellcolor{ptctitle!50}$3$ &
\cellcolor{ptcbackground} $4$ &
\cellcolor{ptcbackground} $5$ &
\cellcolor{ptcbackground} $6$ &
\cellcolor{ptcbackground} $7$ &
\cellcolor{ptcbackground} $\cdots$\tabularnewline
\hline\cellcolor{ptctitle!50}$4$ &
\cellcolor{gray!50} $5$ &
\cellcolor{gray!50} $6$ &
\cellcolor{gray!50} $7$ &
\cellcolor{gray!50} $8$ &
\cellcolor{gray!50} $\cdots$\tabularnewline
\hline\cellcolor{ptctitle!50}$\vdots$ &
\cellcolor{ptcbackground} $\vdots$ &
\cellcolor{ptcbackground} $\vdots$ &
\cellcolor{ptcbackground} $\vdots$ &
\cellcolor{ptcbackground} $\vdots$ &
\cellcolor{ptcbackground} $\vdots$\tabularnewline
\end{tabular}

}\label{tun}
\end{table}
Observemos en ta tabla que $1+1=2$ o que $3+1=4$ , que es lo que
conocemos desde la básica primaria.\medskip{}
\medskip{}

\begin{axioma}{ Inducción \label{induccion}}

Si $A\subseteq\na,$ $A\neq\emptyset,$ tal que 
\begin{description}
\item [{i)}] $1\in A.$
\item [{ii)}] Si $\left(n\in A\Rightarrow n+1\in A\right)\Rightarrow A=\na.$
\end{description}
\end{axioma}

El axioma de inducción se puede presentar también de la siguiente
forma :

Sea $P$ una propiedad cualesquiera. entonces todo número satisface
la propiedad $P$ si se tiene que:
\begin{description}
\item [{i)}] $1$ satisface la propiedad $P$, es decir $P\left(1\right)$es
cierto.
\item [{ii)}] Si $n$ satisface la propiedad $P$, entonces $n+1$ también
satisface la propiedad $P$, es decir 
\[
\mbox{Si \ensuremath{P\left(n\right)\Rightarrow P\left(n+1\right).}}
\]
\item [{iii)}] Se concluye $P$ se satisface para cualquier natural, es
decir $P\left(m\right)$ es cierta $\forall m\in\na.$ 
\end{description}
\selectlanguage{english}%
\begin{apunte}

\selectlanguage{spanish}%
Cuando el axioma se presenta de esta forma se le llama Principio de
Inducción. 

\end{apunte}

Resumiendo lo que afirman estos axiomas, podemos entender que se trata
de un conjunto que tiene un elemento mínimo, el uno (Axioma \myref{ap4}),
es decir no es siguiente de ningún otro (Axioma \myref{ap2}.i), es
decir, se trata del primer elemento del conjunto, y todos los demás
elementos tienen cada uno un elemento siguiente (Axioma \myref{induccion}.ii),
de modo que dos elementos distintos tienen siguientes distintos. El
axioma \myref{induccion} es de suma importancia por dotarnos de un
método de demostración de propiedades, ya que nos indica que todo
conjunto $A$ al que pertenezca el uno, y tal que todo elemento de
$A$ tiene siguiente en $A$, necesariamente ha de coincidir con el
conjunto $\na$ de los números naturales. Es lo que se acostumbra
a denominar método simple de inducción completa. 

A partir de estos cinco axiomas, y usando sistemáticamente el axioma
\myref{induccion}, o de la inducción completa, podemos probar todas
las propiedades del conjunto $\na$. 

\begin{teo}{}{teo1}

Ningún número natural coincide con su sucesor, es decir $\forall n\in\na,\:n\neq\varphi\left(n\right).$

\end{teo}

Este teorema nos da la herramienta para establecer que el conjunto
de los números naturales no tiene elemento máximo, ya que todo número
natural tiene sucesor de acuerdo con el axioma \myref{ap5} y de acuerdo
con el teorema \ref{th:teo1} ningún natural coincide con su sucesor.

\begin{defi}{Cero}{cero} 

Existe un número llamado neutro aditivo o cero que representa con
el símbolo $"0$'', tal que si $a\in\na\Rightarrow a+0=0+a=a$, además
definiremos la aplicación 
\[
\varphi^{*}\left(n\right)=\begin{cases}
\varphi\left(n\right) & \mbox{si \ensuremath{n\in\na,}}\\
1 & \mbox{si \ensuremath{n=0.}}
\end{cases}
\]

\end{defi}

Observe que esta definición no contradice el axioma \myref{ap2}i
ya que $0\notin\na$ y también nos muestra que el conjunto de los
números naturales lo podemos definir estableciendo un elemento como
mínimo y construyendo el resto con la función sucesor por lo tanto
podemos establecer ya sea el $0$ o el $1$ como el mínimo del conjunto
y a partir de este construir el resto.

Hasta aquí podemos ver que la idea intuitiva de numero está de acuerdo
con la construcción que hicimos, ya que el conjunto se construyó contando
de uno en uno a partir del $1$, como lo habíamos comentado en la
introducción.

\begin{apunte} 

Podemos ahora construir un nuevo conjunto $\na_{0}:=\na\cup\left\{ 0\right\} $
y podemos redefinir la adición 

\begin{center}
\begin{tabular}{cccc}
$+$ : &
$\na_{0}\times\na_{0}$ &
$\rightarrow$ &
$\na_{0}$\tabularnewline
 &
$\left(a,b\right)$ &
$\mapsto$ &
$c=a+b$\tabularnewline
\end{tabular}
\par\end{center}

Si definimos la siguiente proposición $0:=0+0$. Con esta definición
acabamos con la discusión si el cero es o no un número natural, para
nosotros no, pero cuando lo necesitemos trabajamos con el conjunto
$\na_{0}$ 

El axioma de inducción matemática lo podemos extender a $\na_{0}$
cambiando la primera condición a $0\in A.$ 

En general el axioma es válido para cualquier subconjunto de $\text{\na\ }$que
le falten un numero finito de los primeros números, es decir si $B_{0}\subset\na$
tal que $B_{0}=\{a_{0},a_{1},\cdots,\},\:a_{i}\in\na,\;\forall i=0,1,2,\cdots$,
entonces el axioma de inducción se cumple cambiando la primara condición
por $a_{0}\in A.$

\end{apunte}

\subsection{Propiedades de la adición entre números naturales}

\begin{teo}{Propiedades de la adición}{ps}

$\forall a,b,c\in\na_{0}$ se cumplen las siguientes propiedades:
\begin{description}
\item [{i)}] Propiedad asociativa: $\left(a+b\right)+c=a+\left(b+c\right).$
\item [{ii)}] Propiedad conmutativa: $a+b=b+a$. 
\item [{iii)}] Propiedad cancelativa: Si $a+c=b+c\Rightarrow a=b.$
\end{description}
\end{teo}
