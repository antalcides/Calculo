\documentclass{beamer}

\usepackage[utf8]{inputenc}

% Revisar la página A Beamer Quickstart: http://userpages.umbc.edu/~rostamia/beamer/

% Configuracion básica de beamer
\usetheme{Madrid}       % intente Darmstadt, Madrid, Warsaw, Boadilla, ...
\usecolortheme{default} % intente albatross, beaver, crane, ...
% o use el paquete structure para colocar el código RGB
%\usecolortheme[RGB={205,176,0}]{structure}
\usefonttheme{default}  % intente serif, structurebold, ...
\setbeamertemplate{navigation symbols}{}
\setbeamertemplate{caption}[numbered]

% Cambia la combinación de fondo de todas las láminas, 
% descomente para ver su efecto, cambie los colores
%\setbeamertemplate{background canvas}[vertical shading][bottom=red!20,top=yellow!30]

% Cambia el color de las fuentes de todo el documento, decomente para probar
%\setbeamercolor{normal text}{fg=green!30!black}

% Coloca rejilla en la lámina, descomente y/o cambie valores
%\setbeamertemplate{background}[grid][step=4mm,color=gray]

% Controla la transparencia del texto al usar \pause
\setbeamercovered{transparent=20} % default es 15%

% Iconos, comente y descomente para ver cambios
%\setbeamertemplate{items}[ball]
%\setbeamertemplate{items}[circle]
\setbeamertemplate{items}[triangle]

%Otros paquetes 
\usepackage[spanish]{babel} % Etiquetas y fecha en castellano
\usepackage{ragged2e}       % Paquete para justificar texto con el uso de \justifying
\usepackage{pifont}         % Paquete para usar íconos pifont, buscar tabla pifont en internet
\usepackage{tcolorbox}

\hypersetup{colorlinks=true,linkcolor=red} % Coloca los enlaces en rojo

\title[Beamer]{Modelo de presentación en Beamer}

\subtitle[Presentación]{Presentación corta}

\author[P. Pérez]{Pedro Pérez}

\institute[FACYT-UC]{
  Departamento de Física\\
  Facultad Experimental de Ciencias y Tecnología\\
  Universidad de Carabobo\\
  \texttt{pperez@uc.edu}
} 

\date[1815]{Octubre de 1815} % Comentar si desea la fecha automática

\begin{document}

% Lámina para la página de título
{\usebackgroundtemplate{\includegraphics[width=\paperwidth]{fondo.jpg}}
% Con el comando anterior puede cambiar la imágen de fondo de una lámina
\begin{frame}[plain]
 \titlepage
\end{frame}
}

%%%%%%%%%%%%%%%%%%%%%%%%%%%%%%%%%%%%%%%%%%%%%%%%%%%%%%%%%%%%%%%%%

% Lámina 1
\begin{frame}{Teorema y entorno de columnas}

Se recomienda revisar la página web \href{http://userpages.umbc.edu/~rostamia/beamer/}{A Beamer Quickstart}

% Caja para teorema
\begin{theorem}[La desigualdad de Poincaré]\justifying
 Suponga que $\Omega\in\mathbf{R}^n$ es un dominio delimitado con fronteras
 suaves. Entonces existe un $\lambda>0$, que depende solamente de $\Omega$,
 tal que para cualquier función $f$ en el espacio de Sobolev $H^1_0(\Omega)$
 se tiene
\[
  \int_\Omega |\nabla u|^2 \,dx \ge 
  \lambda \int_\Omega |u|^2 \,dx .
\]
\end{theorem}

Aquí los entornos de lista \emph{itemized} y el \emph{enumerated} lucen como:

\begin{columns} % Inicio del entorno de columnas
 % Puede colocar 2 o más columnas
  \begin{column}{0.40\textwidth} % Primera columna
  \begin{itemize}
    \item 1er ítem de la lista
    \item 2do ítem de la lista
    \item 3er ítem de la lista
  \end{itemize}
  \end{column}
  
  \begin{column}{0.40\textwidth} % Tercera columna
  \begin{enumerate}
    \item Primer ítem de la lista
    \item Segundo ítem de la lista
    \item Tercer ítem de la lista
  \end{enumerate}
  \end{column}
  
\end{columns}% Fin del entorno de columnas

\end{frame}

%%%%%%%%%%%%%%%%%%%%%%%%%%%%%%%%%%%%%%%%%%%%%%%%%%%%%%%%%%%%%%%%%

% Lámina 2
{\setbeamercolor{background canvas}{bg=cyan!50!white}
% El comando anterior cambia el fondo de la lámina
\begin{frame}{Cambiando el color de fondo}
 
 Utilice {\tt \textbackslash setbeamercolor\{background canvas\}\{bg=color\} } para cambiar el color de fondo de la lámina.
 
\end{frame}
}

%%%%%%%%%%%%%%%%%%%%%%%%%%%%%%%%%%%%%%%%%%%%%%%%%%%%%%%%%%%%%%%%%

% Lámina 3

\begin{frame}{Pausar la aparición del contenido} 
 
In this talk I will give a very elementary proof of the 
theorem.  I am surprised that no one else has thought of 
this before. 
\medskip % Espacio entre parágrafos
 
% La transparencia del texto se controla con  
% \setbeamercovered{transparent=20} al comienzo del archivo.
\pause 
 
Fermat's Last Theorem says that the equation 
\[ 
  x^2 + y^2 = z^2 
\] 
has no solution in the set of natural numbers. 
\medskip 
 
\pause 
 
This is not true.  After a lengthy calculation on the 
department's Linux machines, I have verified that within 
the numerical accuracy of the Pentium-4 processor, we have: 
\[ 
  5000^2 + 12000^2 = 13000^2 
\] 
 
\end{frame} 

%%%%%%%%%%%%%%%%%%%%%%%%%%%%%%%%%%%%%%%%%%%%%%%%%%%%%%%%%%%%%%%%%

% Lámina 4

\begin{frame}[bottom]{Colores y cajas}
 
 \textcolor{blue}{This text is in blue}
 
 \vspace{0.5cm}
 
 \colorbox{yellow}{This text is highlighted in yellow}
 
 \vspace{0.5cm}
 
 \colorbox{yellow}{ 
    \textcolor{red}{ 
        \textbf{ 
            Bold text in red, highlighted in yellow 
        } 
    } 
 } 
 
 \vspace{0.5cm}
 
 \fcolorbox{red}{yellow}{A yellow box with red border}
 
 \vspace{0.5cm}
 
 \setlength{\fboxrule}{4pt} 
 \fcolorbox{red}{white}{A white box with a red border of thickness 4 points}
 
 \vspace{0.5cm}
 
 \setlength{\fboxrule}{4pt} 
\setlength{\fboxsep}{5pt} 
\fcolorbox{red}{white}{A white box with a red border and separation of 5 points}
 
\end{frame}

%%%%%%%%%%%%%%%%%%%%%%%%%%%%%%%%%%%%%%%%%%%%%%%%%%%%%%%%%%%%%%%%%

% Lámina 5

\begin{frame}{Cajas de colores de Beamer}
  
  Esta es una ecuación \footnote{Esta es una referencia}
  
\begin{block}{block}
Esta es una ecuación $x^2$
\[
x_{t+1} = (1-\epsilon)f(x_t)+\epsilon ~ f(y_t),
\]
\end{block}

\begin{exampleblock}{exampleblock}
Esta es una ecuación
\[
x_{t+1} = (1-\epsilon)f(x_t)+\epsilon ~ f(y_t),
\]
\end{exampleblock}

\begin{alertblock}{alertblock}
Esta es una ecuación
\[
x_{t+1} = (1-\epsilon)f(x_t)+\epsilon ~ f(y_t),
\]
\end{alertblock}
 
\end{frame}

%%%%%%%%%%%%%%%%%%%%%%%%%%%%%%%%%%%%%%%%%%%%%%%%%%%%%%%%%%%%%%%%%

% Lámina 6

\begin{frame}{Más cajas de colores}

Modelo de caja con el paquete {\tt tcolorbox}

\begin{tcolorbox}[colback=blue!5,colframe=blue!75!black,title=My title]
  My cool formalization
  \tcblower
  $\displaystyle\sum\limits_{i=1}^n i = \frac{n(n+1)}{2}$
\end{tcolorbox}

Caja sin título

\begin{tcolorbox}[colback=blue!5,colframe=blue!75!black]
  My cool formalization
  \tcblower
  $\displaystyle\sum\limits_{i=1}^n i = \frac{n(n+1)}{2}$
\end{tcolorbox}

Revisar el manual de tcolorbox en \href{http://get-software.net/macros/latex/contrib/tcolorbox/tcolorbox.pdf}{Tcolorbox}

\end{frame}

%%%%%%%%%%%%%%%%%%%%%%%%%%%%%%%%%%%%%%%%%%%%%%%%%%%%%%%%%%%%%%%%%

% Lámina 7

\begin{frame}[label=regreso]{Pifonts y transdissolve}

%\usepackage{pifont}

Uso de transdissolve en entorno itemize acompañado de íconos del paquete {\tt pifont}

%\transdissolve<1-4>
\begin{itemize}\transdissolve<1-4>
 \item<1->[\ding{165}] This is a item
 \item<2->[\ding{34}] This is a item
 \item<3->[\ding{114}] This is a item
 \item<4->[\ding{96}] This is a item
\end{itemize}
Para ver la tabla de códigos para pifont ir al enlace \href{http://willbenton.com/wb-images/pifont.pdf}{Pifont}

\vspace{0.2cm}% Espacio vertical añadido

{\bf \large Páginas enlazadas para hacer salto entre ellas}

Si haces click en la frase \hyperlink{Objetivo}{Página objetivo}, salta para la lámina etiquetada como Objetivo.

Otra forma es con el uso de botones:

\hyperlink{Objetivo}{\beamerbutton{Objetivo}}

Otros botones (sin enlace por los momentos) son:
\beamerbutton{here} 
\beamergotobutton{here} 
\beamerskipbutton{here} 
\beamerreturnbutton{here}

\end{frame}

%%%%%%%%%%%%%%%%%%%%%%%%%%%%%%%%%%%%%%%%%%%%%%%%%%%%%%%%%%%%%%%%%

% Lámina 8

\begin{frame}[label=Objetivo]{Objetivo: lugar de salto}

Esta lámina es etiquetada como Objetivo, Ahora me regreso a la página de origen

\hyperlink{regreso}{Pifont},  o regreso por \hyperlink{regreso}{\beamerreturnbutton{Pifont}}

\end{frame}

%%%%%%%%%%%%%%%%%%%%%%%%%%%%%%%%%%%%%%%%%%%%%%%%%%%%%%%%%%%%%%%%%

% Lámina 9

\begin{frame}{Secciones, subseciones y figuras}

Se puede agrupar el contenido de las láminas con {\tt \textbackslash section}  y {\tt \textbackslash subsection} para
generar tabla de contenidos y menús interactivos en algunos temas de Beamer como {\tt Warsaw}.

\vspace{0.2cm} % Coloca distancia vertical de 0.2 cm

% Ejemplo para colocar una línea horizontal del largo del texto
\noindent\rule{\textwidth}{0.4pt}

\vspace{0.2cm} % Coloca distancia vertical de 0.2 cm

{\tt \textbackslash section}  y {\tt \textbackslash subsection} deben colocarse fuera del entorno {\tt \textbackslash begin\{frame\} \textbackslash end\{frame\}. }

\vspace{0.2cm} % Coloca distancia vertical de 0.2 cm

% Ejemplo para colocar una línea horizontal de 2cm de largo
\noindent\rule{2cm}{0.4pt}

\vspace{0.2cm} % Coloca distancia vertical de 0.2 cm

Se pueden incluir figuras con {\tt \textbackslash includegraphics} incluso dentro de los entornos multicolumnas {\tt \textbackslash columns}

\vspace{0.2cm} % Coloca distancia vertical de 0.2 cm

% Ejemplo para colocar una línea horizontal gruesa del largo del texto
\noindent\rule{\textwidth}{0.8pt}

\vspace{0.2cm} % Coloca distancia vertical de 0.2 cm

También, tal como un árticulo se pueden numerar las ecuaciones y las figuras con el entorno {\tt equation} y con {\tt figure} respectivamente. Sin embargo ésto no es apropiado para todo tipo de presentaciones.
 
\end{frame}


\end{document}
